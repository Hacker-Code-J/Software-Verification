% Propositional Logic

\section{Propositional Logic}

\subsection{Syntax}
\begin{itemize}
	\item \textbf{Atom}: basic elements
	\begin{itemize}
		\item truth symbols $\bot$(``false'') and $\top$(``true'')
		\item propositional variables $P,Q,R,\dots$
	\end{itemize}
	\item \textbf{Literal}: an atom $\alpha$ or its negation $\lnot\alpha$.
	\item \textbf{Formula}: a literal or the application of a logical connective (boolean connectives) to formulas
	\[
	\begin{array}{ccll}
		F & \to & \bot \\
		& | & \top \\
		& | & P \\
		& | & \lnot F & \text{negation (``not'')} \\
		& | & F_1\land F_2 & \text{conjunction (``and'')} \\
		& | & F_1\lor F_2 & \text{disjunction (``or'')} \\
		& | & F_1\to F_2 & \text{implication (``implies'')} \\
		& | & F_1\leftrightarrow F_2 & \text{iff (``if and only if'')}
	\end{array}
	\]
	\item[]
	\item Formula $G$ is a \textbf{subformula} of formula $F$ if it occurs syntactically within $G$.
	\begin{align*}
		\sub{\bot} &= \set{\bot} \\
		\sub{\top} &= \set{\top} \\
		\sub{P} &= \set{P} \\
		\sub{\lnot F} &= \set{\lnot F}\cup\sub{F} \\
		\sub{F_1\land F_2} &= \set{F_1\land F_2}\cup\sub{F_1}\cup\sub{F_2}
		&\vdots
	\end{align*}
	\item Consider $F:(P\land Q)\to (P\lor\lnot Q)$. Then \[
	\sub{F}=\set{F,P\land Q,P\lor\lnot Q,P,Q,\lnot Q}.
	\]
	\item The strict subformulas of a formula are all its subformulas except itself.
	\item[]
	\item To minimally use parentheses, we define the relative precedence of the logical connectives from highest to lowest as follows: \[
	\lnot\quad\land\quad\lor\quad\rightarrow\quad\leftrightarrow
	\]
	\item (Currying) Additionally, $\rightarrow$ and $\leftrightarrow$ associate to the right, e.g., \[
	P\to Q\to R\iff P\to(Q\to R).
	\]
	\item Example:
	\begin{itemize}
		\item $(P\land Q)\to (P\lor\lnot Q)\iff P\land Q\to P\lor \lnot Q$
	\end{itemize}
\end{itemize}