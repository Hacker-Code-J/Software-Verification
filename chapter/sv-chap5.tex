% First-Order Logic

\section{Introduction to FOL}
\begin{itemize}
	\item An extension of propositional logic with predicates, functions, and quantifiers.
	\item First-order logic is also called predicate logic, first-order predicate calculus, and relational logic.
	\item First-order logic is expressive enough to reason about programs.
	\item However, completely automated reasoning is not possible.
\end{itemize}

\subsection{Terms (Variables, Constants, and Functions)}
\begin{itemize}
	\item Terms denote the objects that we are reasoning about.
	\item While formulas in \textsf{PL} evaluate to true or false, term in \textsf{FOL} evaluate to values in an underlying domain such as integers, strings, lists, etc.
	\item Terms in \textsf{FOL} are defined by the grammar: \[
	t\to x\mid c\mid f(t_1,\dots, f_n).
	\] \begin{itemize}
		\item Basic terms are \textbf{variables} $(x,y,z,\dots)$ and \textbf{constant} $(a,b,c,\dots)$.
		\item Composite terms include $n$-ary \textbf{functions} applied to $n$ terms, \ie, $f(t_1,\dots,d_n)$, where $t_1$s are terms.
		\begin{itemize}
			\item A constant can be viewed as a $0$-ary (null-ary) function.
		\end{itemize}
	\end{itemize}
\end{itemize}
\begin{example}
\ \begin{table}[h!]\setstretch{1.25}\centering\begin{tabular}{c|l}
	\hline
	Symbol & \multicolumn{1}{c}{Meaning} \\ \hline
	$f(a)$ & A unary function $f$ applied to a constant \\
	$g(x,b)$ & A binary function $g$ applied to a variable $x$ and a constant $b$ \\
	$f(g(x,f(b)))$ &  \\ \hline
\end{tabular}
\end{table}
\end{example}

\subsection{Predicates}
\begin{itemize}
	\item The propositional variables of \textsf{PL} are generalized to \textbf{predicates} in \textsf{FOL}, denoted $p,q,r,\dots$.
	\item An $n$-ary predicates take $n$ terms as arguments.
	\item A \textsf{FOL} propositional variable is $0$-ary predicate, denoted $P,Q,\dots$.
\end{itemize}
\begin{example}
\ \begin{table}[h!]\setstretch{1.25}\centering\begin{tabular}{c|l}
	\hline
	Symbol & \multicolumn{1}{c}{Meaning} \\ \hline
	$P$ & A propositional variable (or $0$-ary predicate) \\
	$p(f(x),g(x,f(x)))$ & A binary predicate applied to two terms\\ \hline
\end{tabular}
\end{table}
\end{example}

\section{Syntax and Semantics of FOL}
\subsection{Syntax}
\begin{itemize}
	\item \textbf{Atom}: basic elements
	\begin{itemize}
		\item truth symbols $\bot$(``false'') and $\top$(``true'')
		\item $n$-ary predicates applied to $n$ terms.
	\end{itemize}
	\item \textbf{Literal}: an atom $\alpha$ or its negation $\lnot\alpha$.
	\item \textbf{Formula}: a literal or the application of a logical connective to formulas or the application of a quantifier to a formula.
	\[
	\begin{array}{ccll}
		F & \to & \bot\mid\top\mid p(t_1,\dots,t_n) & \text{atom}\\
		& | & \lnot F & \text{negation (``not'')} \\
		& | & F_1\land F_2 & \text{conjunction (``and'')} \\
		& | & F_1\lor F_2 & \text{disjunction (``or'')} \\
		& | & F_1\to F_2 & \text{implication (``implies'')} \\
		& | & F_1\leftrightarrow F_2 & \text{iff (``if and only if'')} \\
		& | & \exists x.F[x] & \text{existential quantification}\\
		& | & \forall x.F[x] & \text{universal quantification}
	\end{array}
	\]
\end{itemize}
\begin{remark}[Notation on Quantification]
\ \begin{itemize}
	\item In $\forall x.F[x]$ and $\exists x.F[x]$, $x$ is the \textbf{quantified variable} and $F[x]$ is the \textbf{scope} of the quantifier. We say $x$ in $F[x]$ is \textbf{bound}.
	\item $\forall x.\forall y.F[x,y]$ is often abbreviated by $\forall x,y.F[x,y]$.
	\item The scope of the quantified variable extends as far as possible: e.g., \[
	\forall x.\underbrace{p(f(x),x)\to(\exists y.\underbrace{p(f(g(x,y)),g(x,y))})\land q(x,f(x))}.
	\]
	\item A variable is \textbf{free} in $F[x]$ if it is not bound. $\free(F)$ and $\bound(F)$ denote the free and bound variables of $F$, respectively. A formula $F$ is \textbf{closed} if $F$ has no free variables. E.g., \[
	\forall x.p(f(x),y)\to\forall y.p(f(x),y).
	\]
	\item If $\free(F)=\set{x_1,\dots, x_n}$, then its \textbf{universal closure} is $\forall x_1\dots\forall x_n.F$ and its \textbf{existential closure} is $\exists x_1\dots\exists x_n.F$ They are usually written $\forall *.F$ and $\exists*.F$.
\end{itemize}
\end{remark}
\begin{example}[FOL Formulas]
\ \begin{itemize}
	\item Every cat has its day. \[
	\forall x.cat(x)\to\exists y.day(y)\land itsDay(x,y)
	\]
	\item 
	\item 
	\item Fermat's Last Theorem
\end{itemize}
\end{example}

\subsection{Interpretation}
A \fol \textbf{interpretation} $I:(D_I,\alpha_I)$ is a pair of a domain and an assignment.

\section{Satisfiability and Validity}

\section{Substitution}

\section{Normal Forms}

\section{Soundness, Completeness, and Decidability}

\section{First-Order Theories}