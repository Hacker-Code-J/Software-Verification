\usepackage{kotex}
\usepackage{commath}
\usepackage{multirow}
\usepackage{multicol}
\usepackage{arydshln} % Include this package
\usepackage{bbding}

\usepackage{tikz}
\usepackage{tikz-cd}
\usetikzlibrary{math} % for calculations
\usetikzlibrary{shapes, arrows.meta, positioning, shadows}
\usetikzlibrary{arrows,shapes.geometric}  % Optional, based on your needs
\usetikzlibrary{3d, calc}
\usetikzlibrary{matrix, positioning, arrows.meta, shapes.multipart, chains}
\usetikzlibrary{decorations.pathreplacing,calligraphy}
\usetheme[progressbar=frametitle]{metropolis}
\usepackage{appendixnumberbeamer}

\usepackage{adjustbox}
\usepackage{booktabs}
\usepackage[scale=2]{ccicons}

\usepackage{tcolorbox}

\usepackage{pgfplots}
\usepgfplotslibrary{fillbetween}
\pgfplotsset{compat=1.15}
\usepgfplotslibrary{dateplot}

\usepackage{xspace}
\newcommand{\themename}{\textbf{\textsc{metropolis}}\xspace}

\usepackage[linesnumbered,ruled]{algorithm2e}
\usepackage{algpseudocode}
\usepackage{setspace}
\SetKwComment{Comment}{/* }{ */}
\SetKwProg{Fn}{Function}{:}{end}
\SetKw{End}{end}
\SetKw{DownTo}{downto}

% Define a new environment for algorithms without line numbers
\newenvironment{algorithm2}[1][]{
	% Save the current state of the algorithm counter
	\newcounter{tempCounter}
	\setcounter{tempCounter}{\value{algocf}}
	% Redefine the algorithm numbering (remove prefix)
	\renewcommand{\thealgocf}{}
	\begin{algorithm}
	}{
	\end{algorithm}
	% Restore the algorithm counter state
	\setcounter{algocf}{\value{tempCounter}}
}

\usepackage{xcolor}   % Required for specifying colors

\usepackage{listings} %Code
\renewcommand{\lstlistingname}{Code}%
\definecolor{keyword}{RGB}{255, 0, 0}
\definecolor{identifier}{RGB}{0, 0, 255}
\definecolor{comment}{RGB}{0, 128, 0}
\definecolor{string}{RGB}{163, 21, 21}

\lstdefinestyle{c}{
	language=C,
	basicstyle=\ttfamily\small,
	keywordstyle=\color{keyword},
	identifierstyle=\color{identifier},
	commentstyle=\color{comment}\itshape,
	stringstyle=\color{string},
	showstringspaces=false,
	%	numberstyle=\tiny\color{gray},
	%	numbersep=5pt,
	frame=single,
	tabsize=4,
	captionpos=b,
	breaklines=true,
	breakatwhitespace=true,
	%	numbers=left
}

\usepackage{amsthm, amsmath}


\newcommand{\cyclic}[1]{\langle #1 \rangle}
\newcommand{\uniform}{\overset{\$}{\leftarrow}}
\newcommand{\N}{\mathbb{N}}
\newcommand{\Z}{\mathbb{Z}}
\newcommand{\Q}{\mathbb{Q}}
\newcommand{\R}{\mathbb{R}}
\newcommand{\C}{\mathbb{C}}
\newcommand{\F}{\mathbb{F}}

\newcommand{\xmark}{\textcolor{red}{\XSolidBrush}}
\newcommand{\vmark}{\textcolor{green!75!black}{\CheckmarkBold}}
